\section{Estado del Arte}

%Lo m\'as importante que se ha hecho hasta ahora con relaci\'on al problema. Deber\'ia responder preguntas como las siguientes:
%?`cuando surge?, ?`qu\'e m\'etodos se han usado para resolverlo?, ?`cuales son los mejores algoritmos que se han creado hasta
%la fecha?, ?`qu\'e representaciones han tenido los mejores resultados?, ?`cu\'al es la tendencia actual?, tipos de movimientos,
%heur\'isticas, m\'etodos completos, tendencias, etc... Puede incluir gr\'aficos comparativos, o explicativos.\\

\textit{Route selection for emergency logistics management} es uno de los problemas fundamentales de logística. En los recientes años, los frecuentes desastres naturales ha incentivado la investigación en el área. El primer trabajo sobre enrutamiento de vehículos fue propuesto por \cite{dantzig1959truck} donde se optimiza el reparto de combustible usando camiones que distribuyen desde una terminal hasta multiples estaciones de servicios. En \cite{ozdamar2004emergency} se propone un modelo para el diseño de rutas en caso emergencias, el objetivo es minimizar la demanda insatisfecha en la ruta planeada. El plan logístico de emergencia incluye los puntos óptimos donde recoger y entregar los materiales en las rutas, estas rutas son regeneradas mientras nuevos materiales y modos de transportes se vuelven disponibles, pero este trabajo no considera que el tiempo de viaje puede variar según el efecto de la catástrofe sobre los caminos.
Según \cite{farahmand1997application,tufekci1995integrated} es importante considerar el efecto de la catástrofe sobre los caminos dado que las condiciones de viaje entre los nodos pueden verse fuertemente afectadas por la extensión del desastre, especialmente en desastres como huracanes e inundaciones que se extienden en tiempo y espacio. Otros trabajos como \cite{Yuan20091081} construyeron un modelo para expresar el efecto de la extensión de un desastre en los caminos, el efecto es representado con la variación de la velocidad de viaje. Para la resolución de este problema se utilizan dos técnicas: la primera basada en el algoritmo de Dijkstra, donde la idea es encontrar el camino más corto paso a paso y la técnica es un algoritmo basado en \textit{Ant system} sin mostrar valores de los parámetros. Los resultados experimentales fueron realizados con 20 nodos evaluando las dos técnicas: Dijkstra y ACO, donde ACO obtuvo mejores resultados en comparación Dijkstra. Además ACO obtuvo las mejores rutas posibles de la instancia sin especificar el tiempo de ejecución.\\
En \cite{zhang2013route} se propone un algoritmo inspirado en la biología observando y modelando el comportamiento de las ameboides para calcular las rutas de emergencia donde la velocidad de los arcos es variable. Los resultados experimentales fueron realizados con 20 nodos donde se obtuvieron las mejores rutas de las instancias sin especifica el tiempo de ejecución.\\
Por otra parte, investigaciones relacionadas \cite{southworth1991regional,cova2003network} han mostrado que las mayores congestiones de un desastre son producidas en las intersecciones de dos arcos en una ruta de emergencia, produciendo un nuevo problema \textit{Lane-based routing} donde la estrategia busca eliminar los cruces en las intersecciones. 
Además, en \cite{xie2011lane} se propone un método para resolver una evacuación considerando el cruce de intersecciones (\textit{lane-based}), la técnica utilizada es tabú search y relajación lagrangiana, a diferencia de \textit{route selection for emergency logistics management} modela el número de vías en un camino (camino con varias vías), el flujo de evacuación de los caminos, entre otros. El algoritmo es aplicado en un red de 100 nodos modelando la evacuación Monticello, Minnesota. El algoritmo converge entre las 350 y 450 iteraciones.
En \cite{stepanov2009multi} se propone un modelo de programación entera para asignar la ruta óptima en caso de evacuación modelando la capacidad de los refugios donde las personas serán evacuadas, además modela la posibilidad de bloqueo por congestión. El algoritmo es evaluado en una instancia con tres áreas a evacuar, la red se encuentra compuesta de 31 nodos.
En \cite{tan2011if} entrega rutas de evacuación incluyendo planes para la distribución de vehículos considerando: las capacidades, emisiones de gases, costos económicos, entre otros. Estos son trabajados por un modelo difuso de evaluación basado en parámetros. Las simulaciones son realizadas con 8 nodos. Dada la incertidumbre y complejidad del ambiente en la emergencia, estos modelos tienen serias limitaciones en manejar el proceso de evaluación basado en los comportamientos individuales \cite{liu2016evacuation}.\\

En \cite{rahman2007feasible} describe una emergencia considerando múltiples aspectos: minimizar el camino de salida en un edificio con múltiples salidas y niveles, utiliza \textit{Ant Colony System} modificado para encontrar una ruta factible para resolver el problema. Define distintos tipos de agentes según el sector: agente señal de salida, agente de pasillo, agente de escaleras y agente habitante. La evaluación es realizada en uno edificio de la Universidad Tecnológica de Petronas en Malaysia, el edificio tiene cuatro pisos, cada piso con cuatro bloques de piezas y cada bloque tiene doce piezas. La función a minimizar en el tiempo de evacuación $T_{clear}$ y el número de ocupantes seguros. El autor realiza una comparación con el software IntelSign y el algoritmo propuesto. Finalmente el algoritmo obtiene mejores resultados que IntelSign.
En \cite{zong2010multi} se presenta un algoritmo \textit{ant colony system} multi-objetivo para resolver problemas de evacuaciones de automóviles y peatones  considerando los aspectos como congestión, número de automóviles y peatones. El autor considera dos objetivos: minimizar el tiempo total de la evacuación y minimizar el número de arcos en camino total. Los experimentos son realizados con una simulación del estadio \textit{Wuhan Sports Center} representada por 157 nodos dentro del estadio, 319 caminos fuera del estadio y 8 zonas de seguridad. Las parámetros utilizados son: $\alpha=2$, $\beta=3$, $\rho=0.7$, $w=1$, $k=0.5$
Como continuación al trabajo anterior en \cite{zong2010multiflow} se utiliza modelo \textit{multi-ant colony system} para considerar la evacuación de automóviles y peatones. Los experimentos son realizados en las mismas condiciones y mismos parámetros, finalmente el autor concluye que el algoritmo MACS puede obtener mejores soluciones en comparación a ACS.
En \cite{forcael2014ant} se desarrolla un ACO para optimizar evacuaciones de personas en caso de tsunami considerando la seguridad, la calidad y la pendiente de los caminos. La simulación se realiza en la ciudad de Penco Chile donde se representa con 67 nodos y dos zonas de seguridad. El autor hace un análisis para una simulación especifica (desde nodo 16 hacia la zona de seguridad 2), mostrando que el algoritmo alcanza su mejor valor a las 50 iteraciones y se mantiene estable después de 300 iteraciones utilizando 100 hormigas. Luego, realiza una simulación con personas reales para validar el modelo, agrupa dos grupos de personas de un total de 34 adultos jóvenes: el primer grupo posicionado de manera aleatoria sin ninguna información sobre rutas de escape y el segundo grupo posicionado de manera aleatoria con información de las rutas construidas por el algoritmo. El autor concluye que los tiempos del modelo y la realidad no presentan diferencias significativas y por lo tanto el modelo utilizado permite modelar efectivamente una evacuación en caso de tsunami. Realizando una comparación de tiempos entre los grupos existen diferencias significabas a favor del modelo, presentando hasta 7 minutos menos de tiempo. \\
En \cite{ohta2016improved} se propone un algoritmo del \textit{Max-Min Ant System}, que utiliza una nueva feromona llamada desodorante que elimina la feromona común de ACO cuando un camino peligroso ha sido encontrado, el objetivo de nueva feromona es poder entregar un mayor nivel de seguridad. La evaluación se realiza con un mapa de distrito de Shinjuku en Tokyo representada una matriz de 200x200, donde cada celda se categoriza como: Camino, área segura, área peligrosa y área no utilizable (murallas, edificios, etc). El área peligrosa aumenta de acuerdo al porcentaje de personas evaluadas en el lugar. La evaluación se realiza con 500 iteraciones sin información de parámetros, finalmente el autor realiza una comparación con \textit{Ant system} y concluye que el algoritmo propuesto permite evitar a nuevas zonas inseguras.
En \cite{liu2016evacuation} se propone la utilización de \textit{quantum ant colony algorithm (QACA)} utilizando conceptos y reglas de computación cuántica y ACO. El autor modela la evacuación desde una zona de peligro hacia una zona segura, cada zona presenta un número determinado de nodos y cada nodo tiene una capacidad máxima. El autor busca minimiza el tiempo de la evacuación y la densidad de personas en los nodos de la zona de seguridad. La evacuación se realiza en una red ficticia con 50 nodos donde se compara los resultados con una técnica ACO, el autor concluye que los resultados obtenido por el algoritmo propuesto utilizando QACA presentan mejor resultados en comparación a ACO.\\
%En \cite{fang2011hierarchical} se propone la utilización de \textit{} Los experimentos son realizados con una simulación del estadio \textit{Wuhan Sports Center} que se representa 157 nodos dentro del estadio, 319 caminos fuera del estadio y 8 zonas de seguridad.
%evolutivo
En \cite{saadatseresht2009evacuation} se utiliza un algoritmo evolutivo multi-objetivo y sistemas de información geográfica para realizar una estrategia de evacuación. En el modelo se busca maximizar la capacidad de las áreas seguras y minimizar la distancia a ellos utilizando el algoritmo \texttt{NSGA-II} \cite{deb2002fast}. \\
En \cite{izquierdo2009forecasting,zheng2012modeling} se propone el uso de\textit{particle swarm optimization (PSO)} para la evacuación de peatones en áreas con alto trafico, el uso de PSO busca poder utilizar la inteligencia individual y colectiva de los agentes.
En \cite{zong2014conflict} se presenta un modelo \textit{discrete particle swarm optimization (DPSO)} para la evacuación de peatones y vehículos en caso de emergencia realizando un modelado y simulación de los movimientos de evacuación e interacción entre peatones y vehículos.  Los experimentos son realizados utilizando el estadio \textit{Wuhan Sports Center} que se representa con 157 nodos dentro del estadio, 319 caminos fuera del estadio y 8 zonas de seguridad. En la evaluación se realiza una evacuación de 10000 peatones y 1000 vehículos y se compra con ACO y MACO propuesto en \cite{zong2010multi,zong2010multiflow} utilizando los siguientes parámetros para ellos $\alpha=2$, $\beta=3$, $\rho=0.7$, $Q=100$, $k=1$ y el coeficiente de comunicación $\lambda=0.5$ para MACO. Los resultados muestran tiempos para los algoritmos PSO son mejores y el autor concluye que el motivo es debido a los mecanismos de aprendizaje de PSO. Por otro lado, el autor concluye que MACO presenta mejores resultados que ACO.
