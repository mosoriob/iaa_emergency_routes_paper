\section{Introducción}
%arreglado segun comentarios de entregable
En los recientes años, los frecuentes desastres naturales y no naturales han producido una gran daño a la población, por ejemplo en Chile: Terremoto de 2010 en Concepción y tsunami (27F), Gran Incendio de Valparaíso 2014, Inundación del Norte de Chile en 2015 e Incendio Zona Centro/Sur de Chile en 2017, a partir de esto se vuelve interesante observar los procedimientos de logística asociados a este tipo de desastres. La logística es una de las mayores actividades durante y después de la emergencia, el reparto de alimentos, medicamentos y abrigo deben ser entregados desde la zona de almacenamiento al área afectada de la manera más rápida posible. Es por esto que el diseño de rutas en casos de emergencia es un tema interesante de estudiar. \\
En este documento se busca entender las variables que afectan al problema, su historia, consideraciones a tomar y los avances que se han realizado en la literatura. Para luego, en las próximas secciones describir de manera más amplia el problema, sus modelos y sus estrategias de resolución. 
%Finalizando con las secciones relacionadas a la implementación donde se describirá la utilización de \textit{Ant Colony System} para el problema.
%Explicaci\'on del problema que se va a estudiar, en que consiste, cuales son sus variables, restricciones y objetivos de manera general.
%Variantes m\'as conocidas que existen.

