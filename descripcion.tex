\section{Descripción propuesta}
\subsection{Representación}
Notación:
\begin{itemize}
	\item $i$ indice para los nodos en la red
	\item $v_i$ nodo $i$ de la red
 	\item $(i,j)$ arco desde i a j
	\item $x_{ij}$ variable de decisión del modelo, $x_{ij} = 1 $ cuando el arco de los nodos $(v_i,v_j)$ es incluido en el camino, en caso contrario $x_{ij} = 0$
	\item $l_{ij}$ largo del arco desde i a j
	\item $s^{o}_{ij}$  velocidad entre los arcos $(v_i, v_j)$ en condiciones normales
	\item $s_{ij}(t)$ velocidad entre los arcos $(v_i, v_j)$ en el tiempo $t$
	\item $\alpha_{ij}$ parámetro decreciente para manipular la velocidad           
	\item $\beta_{ij}$ parámetro decreciente para manipular la velocidad             
	\item $t_{ij}$ tiempo necesario para viajar entre i y j                      
\end{itemize}

La formulación del \textit{route selection for emergency logistics management} se describe de la siguiente manera:

El objetivo del modelo es minimizar el tiempo empleado en el camino. Las ecuaciones \eqref{7}, \eqref{8} y \eqref{9} son parte de la fórmula de recursión del tiempo total para el camino. En la ecuación \eqref{speed} la función decreciente de la velocidad de viaje en el arco $(v_i,v_j)$. La restricción \eqref{full} asegura un camino factible desde $v_1$ hasta $v_n$,la restricción \eqref{circles} asegura que no existan ciclos y \eqref{full} muestra los dominios de las variables.

\begin{equation}
	 \text{mín} \sum_{i=1}^{n}\sum_{j=1}^{n} t_{ij} \cdot x_{ij}
\end{equation}

\begin{equation}\label{7}
  \int_{t_i}^{t_j} s_{ij}(t)dt = l_{ij}
\end{equation}
\begin{equation}\label{8}
  t_{ij} = t_j - t_i
\end{equation}
\begin{equation}\label{9}
t_1 =0
\end{equation}
\begin{equation}\label{speed}
  s_{ij}(t) = s_{ij}^0 \cdot \alpha_{ij} \cdot e^{-\beta_{ij}\cdot t}
\end{equation}

\begin{equation}\label{full}
	\sum_{\substack{j=1\\
                  j \neq i}}^n x_{ij} - \sum_{\substack{j=1\\
                  j \neq i}}^{n} x_{ji} = \begin{cases}
1 & i=1\\
-1 & i=n \\
0 & eoc
\end{cases}
\end{equation}



\begin{equation}\label{circles}
	\sum_{\substack{j=1\\
                  j \neq i}}^n x_{ij} = \begin{cases}
\leq 1 & i\neq n\\
=0 & i=n 
\end{cases}
\end{equation}

\begin{equation}
	x_{ij} = 0,1;i=1,2,\cdots,n;j=1,2,\cdots,n
\end{equation}
\subsection{Operadores/Movimientos}
La decisión para la selección del siguiente nodo se realiza en base al movimiento  definido por:
\begin{equation}
j = \begin{cases} arg max_{u \in S_{k}(i)} \{ [\tau(i,u)]^\alpha [\eta(i,u)]^\beta\} &\mbox{if } q \leq q_0 \\ 
J & otherwise \end{cases} 
\end{equation}

Donde $J \in J_i^k$ se elige según la probabilidad:

	\begin{equation}
		P_{ij}^k(t) = \frac{[\tau_{ij}(t)]^{\alpha_a}[\eta_{ij}]^{\beta_a}}{\sum_{h \in J_i^k } [\tau_{ij}(t)]^{\alpha_a}[\eta_{ij}]^{\beta_a}}
	\end{equation}
	
Para la sección de siguiente nodo se considera la heurística del tiempo expresada por $\eta = \frac{1}{tiempo\_tour}$.
En caso que el nodo sea un nodo seguro se selecciona el nodo y se finaliza la construcción del camino.
%todo: descripcion de cada proceso
\subsection{Descripción de cada proceso}
\subsection{Pseudocódigo general}
En general, el algoritmo implementado está basado en el algoritmo Ant Colony System \cite{dorigo1997ant}. En \ref{alg} se muestra un pseudocógido del algoritmo \\
\begin{algorithm}[]
\label{alg}

 \KwData{Algoritmo propuesto}
 inicializar\;
 \For{cada arco (i,j)}{
 	$\tau_{ij}(0) = \tau_{0}$
 }
 \For{k=1 hasta n}{
 	Ubique la hormiga $k$ en el primer nodo
 }
 Sea $T+$ el tour más corto encontrado desde el inicio\;
 \For{t=1 hasta el número de iteraciones}{
 \For{k=1 hasta m}{
 \While{no se complete el tour}{
  \eIf{si existe un nodo j $\in$ cl }{
   Elegir un nodo $j$, $j \in j_{i}^{k}$ entre $cl$ nodos de la lista\;
\begin{equation*}
j = \begin{cases} arg max_{u \in J_{i}^{k}} \{ [\tau_{iu}(t))]^\alpha [\eta_{iu}]^\beta\} &\mbox{if } q \leq q_0 \\ 
J & otherwise \end{cases}
\end{equation*}
	Donde $J \in J_i^k$ se elige según la probabilidad\;
	\begin{equation}
		P_{ij}^k(t) = \frac{[\tau_{ij}(t)]^{\alpha}[\eta_{ij}]^{\beta}}{\sum_{h \in J_i^k } [\tau_{ih}(t)]^{\alpha}[\eta_{ih}]^{\beta}}
	\end{equation}
Y donde $i$ es el nodo actual.
%else
   }
   {
   elegir el nodo $j \in J_{i}^{k}$ más cercana.
  }
 Luego se actualiza la feromona\;
\begin{equation}
	t_{ij}(t) = (1-\rho)t_{ij}(t) + \rho \tau_0
\end{equation}
 }
 %while tour
 Calcular el tiempo $T_k$ del tour de la hormiga $k$
 }
 Salvar la mejor solución $T^{+}$ encontrada hasta el momento
 %for hormigas
 
 \For{cada arco (i,j) $\in T^{+}$ }{
\begin{equation}
	t_{ij}(t) = (1-\rho)t_{ij}(t) + \rho \Delta \tau_{ij}(t)	
\end{equation}
	donde $\tau_{ij}(t) = \frac{1}{T^{+}}$
 }
 }
 %for iteraciones
 \caption{Algoritmo ACS.}
\end{algorithm}
\subsection{Listado de parámetros del algoritmo}

A continuación se presenta los parámetros utilizados por el algoritmo.
\begin{itemize}
\item Número de iteraciones: $loops=2000$
\item Tamaño de la lista candidata $cl=15$
\item Número de hormigas $M=[6,88]$
\item Parámetro ACS $\alpha_{ant}=[1.29, 4.23]$
\item Parámetro ACS $\beta_{ant}=[2.5,9.83]$
\item Selección $q_0=[0.08,0.94]$
\item Evaporación $\rho=[0.11,0.97]$
\item Feromona inicial $\tau_{0}=0.1$
\end{itemize}
